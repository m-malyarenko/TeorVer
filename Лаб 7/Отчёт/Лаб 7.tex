%%%%%%%%%%%%%%%%%%%%%%%%%%%%%%%%%%%%%%%%%%%%%%%%%%%%%%%%%%%%%%%%%%%%%%%%%%%%%%%
%                     Отчёт по лабораторной работе №7
%
% Дисциплина: Теория вероятностей и Математическая статистика
%                     
% Название:   Метод максимального правдоподобия. Проверка гипотезы о распределении
%             по критерию хи-квадрат
%
% Выполнил:   Михаил Маляренко
%
% Дата:       12 Янв. 2021
%
%%%%%%%%%%%%%%%%%%%%%%%%%%%%%%%%%%%%%%%%%%%%%%%%%%%%%%%%%%%%%%%%%%%%%%%%%%%%%%%

% HEADER BEGIN 
\documentclass[12pt]{article}
\usepackage[utf8]{inputenc}
\usepackage[russian]{babel}
\usepackage{pscyr}
\usepackage[T2A]{fontenc}
\usepackage{geometry}
\usepackage{graphicx}
\usepackage{multirow}
\usepackage{hhline}
\usepackage{amsmath}
\usepackage{amssymb}
\usepackage{hyperref}
\usepackage{xcolor}

\geometry {	
	a4paper, 
	left   = 20mm, 
	right  = 20mm, 
	top    = 20mm, 
	bottom = 20mm
}

\definecolor{urlcolor}{HTML}{2484BC} 
\definecolor{linkcolor}{HTML}{000000}

\graphicspath{{resource/}}
% HEADER END

% DEFINITIONS BEGIN
\newcommand{\lskip}{\hfill\break}
% DEFINITIONS END

\begin{document}

\begin{titlepage}
	\begin{center}
		\hfill \break
		{\textbf{Санкт-Петербургский политехнический университет Петра Великого}}\\
		\hfill \break
		\textbf{Институт прикладной математики и механики}\\
		 \hfill \break
		\textbf{Кафедра <<Телематика (при ЦНИИ РТК)>>}\\
		\vfill
		\large{\bfseries Отчет по лабораторной работе}\\
		\hfill \break
		\hfill \break
		\hfill \break
		\hfill \break
        \normalsize{\bfseriesМетод максимального правдоподобия. Проверка гипотезы о распределении по критерию хи-квадрат}\\
        \hfill \break
		По дисциплине <<Теория вероятностей и Математическая статистика>>\\
		\hfill \break
		\hfill \break
	\end{center}
 
	\normalsize
	{ 
		\begin{tabular}{lp{2cm}cr}
			Выполнил &&&\\
			Студент гр. 3630201/80101&&\underline{\hspace{1.5cm}}& М. Д. Маляренко\\\\
			Руководитель&&&\\ 
			к.ф.-м.н., доцент && \underline{\hspace{1.5cm}}& А. Н. Баженов \\\\
			&&&<<\underline{\phantom{333}}>>\underline{\phantom{сентября000}}
			2020г.
		\end{tabular}
	}
\vfill

\begin{center} Санкт-Петербург \\2021 \end{center}
\end{titlepage}

\newpage

\setcounter{page}{2}

\begin{flushleft}

\setlength{\parindent}{1cm}

% TABLE OF CONTENTS
\tableofcontents

\newpage

% LIST OF TABLES
\listoftables

\newpage

\section{Постановка задачи}

	Сгенерировать выборку объёмом 100 элементов для нормального распределения $N(0,1)$.
	По сгенерированной выборке оценить параметры $\mu$ и $\sigma$ нормального закона методом максимального правдоподобия.
	В качестве основной гипотезы $H_0$ считать, что сгенерированное распределение имеет вид $N(\hat{\mu}, \hat{\sigma})$, где $\hat{\mu}$ и $\hat{\sigma}$ оценки метода максимального правдоподобия.
	Проверить основную гипотезу, используя критерий согласия $\chi^2$. В качестве уровня значимости взять $\alpha = 0.05$.

    Исследовать точность (чувствительность) критерия $\chi^2$ -- сгенерировать выборки равномерного распределения и распределения Лапласа из 20 элементов. Проверить их на нормальность, то есть проверить, принимает ли критерий $\chi^2$ гипотезу, что элементы этих выборок распределены по закону $N(\hat{\mu}, \hat{\sigma})$.

\newpage

\section{Теория}

	\subsection{Метод максимального правдоподобия}
	$L(x_1,\dots,x_n,\theta)$ --- функция правдоподобия (ФП), рассматриваемая как функция неизвестного параметра $\theta$\cite{theory}\cite{maks}:
	\begin{equation}
		L(x_1,\dots,x_n,\theta) = f(x_1,\theta)\dots f(x_n, \theta).
	\end{equation}
	Оценка максимального правдоподобия:
	\begin{equation}
		\hat{\theta}_{\text{МП}} = \arg\max L(x_1,\dots,x_n,\theta)
	\end{equation}
	Система уравнений правдоподобия (в случае дифференцируемости функции правдоподобия):
	\begin{equation}
		\frac{\partial L}{\partial \theta_k} = 0 \; \text{ или } \; \frac{\partial \ln L}{\partial \theta_k} = 0, \; k = 1,\dots,m.
		\label{difEquation}
	\end{equation}

	\subsection{Проверка гипотезы о законе распределения. Метод $\chi^2$}
	Выдвинута гипотеза $H_0$ о генеральном законе распределения с функцией
	распределения $F(x)$.\\
	\phantom{0}\\
	Рассматриваем случай, когда гипотетическая функция распределения $F(x)$
	не содержит неизвестных параметров.\\
	\phantom{0}\\
	\textbf{Правило проверки гипотезы о законе распределения по методу $\chi^2$.}\cite{theory}
	\begin{enumerate}
		\item Выбираем уровень значимости $\alpha$.

		\item Находим квантиль $\chi^2_{1-\alpha}(k-1)$ распределения $\chi^2$ с $k-1$ степенями свободы порядка $1-\alpha$.

		\item С помощью гипотетической функции распределения $F(x)$ вычисляем
		вероятности $p_i = P(X\in \Delta_i), i=1,\dots,k$.

		\item Находим частоты $n_i$ попадания элементов выборки в подмножества
		$\Delta_i, i = 1,\dots,k$.

		\item Вычисляем выборочное значение статистики критерия $\chi^2$ :
		\[
			\chi^2_B = \sum_{i=1}^{n}\frac{(n_i - np_i)^2}{np_i}.    
		\]

		\item Сравниваем $\chi^2_B$ и квантиль  $\chi^2_{1-\alpha}(k-1)$.
		\begin{itemize}
			\item[$\text{а)}$] Если $\chi^2_B < \chi^2_{1-\alpha}(k-1)$, то гипотеза $H_0$ на данном этапе проверки принимается.
			\item[$\text{б)}$] Если $\chi^2_B \geq \chi^2_{1-\alpha}(k-1)$, то гипотеза $H_0$ отвергается, выбирается одно из альтернативных распределений, и процедура проверки повторяется.
		\end{itemize}
	\end{enumerate}
	Количество интервалов $k$ можно определить с помощью эвристики:
	\begin{equation}
		k \approx 1.72\cdot\sqrt[3]{n}
		\label{k}
	\end{equation}

\newpage

\section{Реализация}

	Расчёты проводились в среде аналитических вычислений Maxima. Были написаны функции для нахождения количества промежутков разбиения с помощью эвристики \ref{k}, нахождения границ промежутков разбиения, вычисления теоретической вероятности на промежутках с помощью встроенной реализации функции нормального распределения, вычисления относительной частоты попадания элементов выборки в промежутки разбиения. Полный текст скрипта представлен в репозитории GitHub.

\newpage

\section{Результаты}

	\subsection{Метод максимального правдоподобия}

		По методу максимального правдоподобия были получены следующие оценки параметров выборки, распределённой по закону $N(0, 1)$:

		$$
			\hat{\mu} \approx -0.035 , \quad \hat{\sigma} \approx 0.992 
		$$

	\subsection{Критерий согласия $\chi^2$ для нормального распределения}

	\begin{itemize}
		\item Количество промежутков $k=8$ (\ref{k})
		\item Уровень значимости $\alpha = 0.05$
		\item Квантиль $\chi^2_{1-\alpha}(k-1) = \chi^2_{0.95}(7) = 14.0671$
	\end{itemize}

	В Таблице \ref{chi_normal} представлены этапы вычисления критерия хи-квадрат для проверки гипотезы о законе распределения.

	\begin{table}[h]
		\begin{center}
			\caption{Вычисление $\chi^2_B$ при проверке гипотезы $H_0$ о законе распределения $N(\hat{\mu}, \hat{\sigma})$ для выборки распределения $N(0, 1)$}
			\begin{tabular}{||c|*{4}{c|}c||} \hhline{|t:=:=:=:=:=:=:t|}
				$i$    & $\Delta_i$         & $n_i$ & $p_i$   & $n_ip_i$ & $\frac{(n_i-np_i)^2}{np_i}$ \\
				\hhline{|:=:=:=:=:=:=:|}
				$1$    & $(-\infty, -1.49)$ & $8$   & $0.070$ & $7.02$   & $0.01$ \\
				\hhline{||-|-|-|-|-|-||}
				$2$    & $[-1.49, -1.05)$   & $9$   & $0.084$ & $8.35$   & $0.04$ \\
				\hhline{||-|-|-|-|-|-||}
				$3$    & $[-1.05, -0.59)$   & $7$   & $0.132$ & $13.16$  & $2.91$ \\
				\hhline{||-|-|-|-|-|-||}
				$4$    & $[-0.59, -0.15)$   & $20$  & $0.169$ & $16.94$  & $0.57$ \\
				\hhline{||-|-|-|-|-|-||}
				$5$    & $[-0.15, 0.3)$     & $18$  & $0.178$ & $17.82$  & $0.00$ \\
				\hhline{||-|-|-|-|-|-||}
				$6$    & $[0.3, 0.75)$      & $17$  & $0.153$ & $15.31$  & $0.18$ \\
				\hhline{||-|-|-|-|-|-||}
				$7$    & $[0.75, 1.2)$      & $13$  & $0.108$ & $10.75$  & $0.45$ \\
				\hhline{||-|-|-|-|-|-||}
				$8$    & $[1.20, \infty)$   & $8$   & $0.106$ & $10.62$  & $0.64$ \\
				\hhline{|:=:=:=:=:=:=:|}
				$\sum$ & --                 & $100$ & $1.000$ & $100.0$  & $4.92 = \chi^2_B$ \\
				\hhline{|b:=:=:=:=:=:=:b|}
			\end{tabular}
		\label{chi_normal}
		\end{center}
	\end{table}

	Табличное значение $\chi^2_{0.95}(7) = 14.0671$ больше, чем выборочное $\chi^2_B = 4.92$, это значит, что на данном этапе гипотезу $H_0$ можно принять.

	\subsection{Проверка чувствительности критерия согласия $\chi^2$}

		\subsubsection{Для равномерного распределения}

			\begin{itemize}
				\item Количество промежутков $k=5$
				\item Уровень значимости $\alpha = 0.05$
				\item Квантиль $\chi^2_{1-\alpha}(k-1) = \chi^2_{0.95}(4) = 9.487$
			\end{itemize}

			В Таблице \ref{chi_uni} представлены этапы вычисления критерия хи-квадрат для проверки гипотезы о законе распределения.

			\newpage

			\begin{table}[h]
				\begin{center}
					\caption{Вычисление $\chi^2_B$ при проверке гипотезы $H_0$ о законе распределения $N(\hat{\mu}, \hat{\sigma})$ для выборки распределения $U(-\sqrt{3}, \sqrt{3})$}
					\begin{tabular}{||c|*{4}{c|}c||} \hhline{|t:=:=:=:=:=:=:t|}
						$i$    & $\Delta_i$         & $n_i$ & $p_i$ & $n_ip_i$ & $\frac{(n_i-np_i)^2}{np_i}$ \\
						\hhline{|:=:=:=:=:=:=:|}
						$1$    & $(-\infty, -0.88)$ & $5$  & $0.171$ & $3.42$ & $0.73$ \\
						\hhline{||-|-|-|-|-|-||}
						$2$    & $[-0.88, -0.18)$   & $2$  & $0.207$ & $4.13$ & $1.10$ \\
						\hhline{||-|-|-|-|-|-||}
						$3$    & $[-0.18, 0.52)$    & $4$  & $0.250$ & $5.00$ & $0.20$ \\
						\hhline{||-|-|-|-|-|-||}
						$4$    & $[0.52, 1.22)$     & $4$  & $0.204$ & $4.09$ & $0.00$ \\
						\hhline{||-|-|-|-|-|-||}
						$5$    & $[1.22, \infty)$   & $5$  & $0.167$ & $3.34$ & $0.82$ \\
						\hhline{|:=:=:=:=:=:=:|}
						$\sum$ & --                 & $20$ & $1.000$ & $20.0$ & $2.86 = \chi^2_B$ \\
						\hhline{|b:=:=:=:=:=:=:b|}
					\end{tabular}
				\label{chi_uni}
				\end{center}
			\end{table}

			Табличное значение $\chi^2_{0.95}(7) = 9.487$ больше, чем выборочное $\chi^2_B = 2.86$, это значит, что на данном этапе гипотезу $H_0$ можно принять.

		\subsubsection{Для распределения Лапласа}

			\begin{itemize}
				\item Количество промежутков $k=5$
				\item Уровень значимости $\alpha = 0.05$
				\item Квантиль $\chi^2_{1-\alpha}(k-1) = \chi^2_{0.95}(4) = 9.487$
			\end{itemize}

			В Таблице \ref{chi_laplace} представлены этапы вычисления критерия хи-квадрат для проверки гипотезы о законе распределения.

			\begin{table}[h]
				\begin{center}
					\caption{Вычисление $\chi^2_B$ при проверке гипотезы $H_0$ о законе распределения $N(\hat{\mu}, \hat{\sigma})$ для выборки распределения $L(0, 1/\sqrt{2})$}
					\begin{tabular}{||c|*{4}{c|}c||} \hhline{|t:=:=:=:=:=:=:t|}
						$i$    & $\Delta_i$         & $n_i$ & $p_i$ & $n_ip_i$ & $\frac{(n_i-np_i)^2}{np_i}$ \\
						\hhline{|:=:=:=:=:=:=:|}
						$1$    & $(-\infty, -0.83)$ & $5$  & $0.184$ & $3.68$ & $0.47$ \\
						\hhline{||-|-|-|-|-|-||}
						$2$    & $[-0.83, -0.31)$   & $3$  & $0.182$ & $2.98$ & $0.00$ \\
						\hhline{||-|-|-|-|-|-||}
						$3$    & $[-0.31, 0.20)$    & $5$  & $0.128$ & $3.64$ & $0.50$ \\
						\hhline{||-|-|-|-|-|-||}
						$4$    & $[0.20, 0.72)$     & $2$  & $0.179$ & $3.58$ & $0.69$ \\
						\hhline{||-|-|-|-|-|-||}
						$5$    & $[0.72, \infty)$   & $5$  & $0.306$ & $6.11$ & $0.20$ \\
						\hhline{|:=:=:=:=:=:=:|}
						$\sum$ & --                 & $20$ & $1.000$ & $20.0$ & $1.88 = \chi^2_B$ \\
						\hhline{|b:=:=:=:=:=:=:b|}
					\end{tabular}
				\label{chi_laplace}
				\end{center}
			\end{table}

			Табличное значение $\chi^2_{0.95}(4) = 9.487$ больше, чем выборочное $\chi^2_B = 1.88$, это значит, что на данном этапе гипотезу $H_0$ можно принять.

\newpage

\section*{Заключение}
\addcontentsline{toc}{section}{Заключение}

В результате выполнения лабораторной работы была изучена оценка параметров распределения по выборке распределённой согласно закону $N(0, 1)$ с помощью метода максимального правдоподобия. Выяснено, что о.м.п. матожидания нормального распределения равна средневыборочному значению, а о.м.п. среднеквадратического отклонения вычисляется как стандартное выборочное отклонение.

Также были проанализированы результаты оценки гипотез о распределения выборочных элементов по критерию хи-квадрат. Для нормального распределения по закону $N(0, 1)$ критерий хи-квадрат не отверг гипотезу о том, что выборка распределена по закону $N(\hat{\mu}, \hat{\sigma})$, где $\hat{\mu}$ - о.м.п. матожидания, $\hat{\sigma}$ -- о.м.п. среднеквадратического отклонения.

Проверка чувствительности критерия хи-квадрат на малых выборках показала, что этот критерий может давать неверные результаты, то есть является не чувствительным на малых выборках, так как он не отверг гипотезу о нормальлности распределения $U(-\sqrt{3}, \sqrt{3})$ и $L(0, 1/\sqrt{2})$
\newpage

\addcontentsline{toc}{section}{Список литературы}
	\begin{thebibliography}{9}

		\bibitem{theory}
		Теоретическое приложение к лабораторным работам №5-8 по дисциплине «Математическая статистика». -- СПб.: СПбПУ, 2020. -- 22 c 
		
		\bibitem{maks}
		Максимов Ю.Д. Математика. Теория и практика по математической статистике. Конспект-справочник по теории вероятностей : учеб. пособие / Ю.Д. Максимов; под ред. В.И. Антонова. — СПб. : Изд-во Политехн. ун-та, 2009. — 395 с. (Математика в политехническом университете).

	\end{thebibliography}

\newpage

\section*{Приложение А. Репозиторий с исходным кодом}
\addcontentsline{toc}{section}{Приложение А. Репозиторий с исходным кодом}

Исходный код скрипта для среды аналитических вычислений Maxima находится в репозитории GitHub -- URL \url{https://github.com/malyarenko-md/TeorVer}

\end{flushleft}

\end{document}