%%%%%%%%%%%%%%%%%%%%%%%%%%%%%%%%%%%%%%%%%%%%%%%%%%%%%%%%%%%%%%%%%%%%%%%%%%%%%%%
%                     Отчёт по лабораторной работе №2
%
% Дисциплина: Теория вероятностей и Математическая статистика
%                     
% Название:   Вычисление выборочных характеристик
%
% Выполнил:   Михаил Маляренко
%
% Дата:       23 Ноя. 2020
%
%%%%%%%%%%%%%%%%%%%%%%%%%%%%%%%%%%%%%%%%%%%%%%%%%%%%%%%%%%%%%%%%%%%%%%%%%%%%%%%

% HEADER BEGIN 
\documentclass[12pt]{article}
\usepackage[utf8]{inputenc}
\usepackage[russian]{babel}
\usepackage{pscyr}
\usepackage[T2A]{fontenc}
\usepackage{geometry}
\usepackage{multirow}
\usepackage{amsmath}
\usepackage{amssymb}
\usepackage{hyperref}
\usepackage{xcolor}

\geometry {	
	a4paper, 
	left   = 20mm, 
	right  = 20mm, 
	top    = 20mm, 
	bottom = 20mm
}

\definecolor{urlcolor}{HTML}{2484BC} 
\definecolor{linkcolor}{HTML}{000000}
% HEADER END

% DIFINES BEGIN
\newcommand{\lskip}{\hfill\break}
% DEFINES END

\begin{document}

\begin{titlepage}
	\begin{center}
		\hfill \break
		{\textbf{Санкт-Петербургский политехнический университет Петра Великого}}\\
		\hfill \break
		\textbf{Институт прикладной математики и механики}\\
		 \hfill \break
		\textbf{Кафедра <<Телематика (при ЦНИИ РТК)>>}\\
		\vfill
		\large{\bfseries Отчет по лабораторной работе}\\
		\hfill \break
		\hfill \break
		\hfill \break
		\hfill \break
		\normalsize{\bfseries Вычисление выборочных характеристик}\\
		По дисциплине <<Теория вероятностей и Математическая статистика>>\\
		\hfill \break
		\hfill \break
		\hfill \break
	\end{center}
 
	\normalsize
	{ 
		\begin{tabular}{lp{2cm}cr}
			Выполнил &&&\\
			Студент гр. 3630201/80101&&\underline{\hspace{1.5cm}}& М.Д. Маляренко\\\\
			Руководитель&&&\\ 
			к.ф.-м.н., доцент && \underline{\hspace{1.5cm}}& А.Н. Баженов \\\\
			&&&<<\underline{\phantom{333}}>>\underline{\phantom{сентября000}}
			2020г.
		\end{tabular}
	}
\vfill

\begin{center} Санкт-Петербург \\2020 \end{center}
\end{titlepage}

\newpage

\setcounter{page}{2}

\begin{flushleft}

\setlength{\parindent}{1cm}

\tableofcontents

\newpage

\listoftables

\newpage

\section{Постановка задачи}

	Заданы 5 распределений случайных величин:

	\begin{enumerate}
		\item Нормальное распределение $N(x, 0, 1)$
		\item Распределение Коши $C(x, 0, 1)$
		\item Распределение Лапласа $L(x, 0, 1/\sqrt{2})$
		\item Дискретное распределение Пуассона $P(k, 10)$
		\item Равномерное распределение $U(x, -\sqrt{3}, \sqrt{3})$
	\end{enumerate}

	\noindentДля каждого распределения необходимо сгенерировать выборки размером 10, 100 и 1000 элементов. Для каждой выборки рассчитать числовые характеристики:

	\begin{enumerate}
		\item Выборочное среднее $\overline{x}$
		\item Выборочная медиана $med \; x$
		\item Полусумма экстремальных выборочных элементов $z_R$
		\item Полусумма квартилей $z_Q$
		\item Усечённое среднее $z_{tr}$
	\end{enumerate}

	Произвести генерацию каждой выборки и вычисление характеристик 1000 раз. Найти среднее характеристик $E(z)$, вычислить дисперсию $D(z)$. 

\newpage

\section{Теория}

	\subsection{Вариационный ряд}

		Вариационным рядом называется последовательность не обязательно уникальных элементов выборки, расположенных порядке неубывания. \cite{theory}
		\lskip

		\noindentЗапись элементов вариационного ряда:

		$$
		x_{(i)}, \; i = 1, \dots, n
		$$

	\subsection{Выборочные числовые характеристики}

		В данной лабораторной работе рассматриваются следующие числовые выборочные характеристики для вариационного ряда $x_{(1)}, \dots, x_{(n)}$. \cite{theory}

			\begin{itemize}
				\item \textit{Выборочное среднее}
				
				\begin{equation}
					\overline{x} = \frac{1}{n} \sum_{i=1}^{n}x_{(i)}
					\label{mean}
				\end{equation}
					
				\item \textit{Выборочная медиана}
				
				\begin{equation}
					med \; x = 
					\left\{
					\begin{aligned}
						& x_{(l + 1)}, \qquad \quad n = 2l + 1\\
						& \frac{x_{(l)} + x_{(l + 1)}}{2}, \; n = 2l\\
					\end{aligned}
					\right.
					\label{med}
				\end{equation}

				\item \textit{Полусумма экстремальных выборочных элементов}
				
				\begin{equation}
					z_R = \frac{x_{(1)} + x_{(n)}}{2}
					\label{extreme}
				\end{equation}

				\item \textit{Полусумма квартилей}
				
				Выборочная квартиль:

				$$
				z_p = 
				\left\{
				\begin{aligned}
					& x_{([np] + 1)}, & np\text{ - целое}\\
					& x_{(np)}, \; & np\text{ - дробное}\\
				\end{aligned}
				\right.
				$$
		
				Полусумма квартилей:

				\begin{equation}
					z_Q = \frac{z_{1/4} + z_{3/4}}{2}
					\label{quart}
				\end{equation}

				\item \textit{Усечённое среднее}
				
				\begin{equation}
					z_{tr} = \frac{1}{n - 2r} \sum_{i=r+i}^{n - r}x_{(i)}, \qquad r \approx \frac{n}{4}  
					\label{trunc}
				\end{equation}

			\end{itemize}

	\subsection{Средее характеристик, дисперсия}
		
		Среднее характеристик рассчитывается как простое среднее арифметическое.

		\begin{equation}
			E(z) = \overline{z}
			\label{E}
		\end{equation}

		Рассеяние или дисперсия рассчитывается как разность среднеквадратичного значения вариационного ряда и квадрата выборочного среднего:

		\begin{equation}
			D(z) = \overline{x^2} - {\overline{x}}^2
			\label{D}
		\end{equation}
		
\newpage

\section{Реализация}

Расчёты были реализованы в среде аналитических вычислений Maxima. Выборки сгенерированы встроенными функциями среды Maxima. Код скрипта представлен в репозитории на GitHub.

\newpage

\section{Результаты}

По результатам вычисления выборочных числовых характеристик были сформированы таблицы 1-5 по количеству заданных распределений. Погрешность среднего значения характеристики выборки рассчитывалась как $\Delta_z = \sqrt{D(z)}$.

	\begin{table}[h]
		\begin{center}
			\begin{tabular}{|*{7}{c|}} \hline
				\multicolumn{7}{|c|}{Нормальное распределение}\\ \hline
				\multicolumn{2}{|c|}{} & $\overline{x} \; (\ref{mean})$ & $med \; x\; (\ref{med})$ & $z_R \; (\ref{extreme})$ & $z_Q\; (\ref{quart})$ & $z_{tr}\; (\ref{trunc})$ \\ \hline
				\multirow{2}*{$N = 10$}   & $E(z) \; (\ref{E}) \pm \Delta_z$ & $0.0 \pm 0.4$ & $0.0 \pm 0.7$ & $0.0 \pm 0.8$ & $0.0 \pm 0.8$ & $0.0 \pm 0.5$ \\ \cline{2-7}
										& $D(z) \; (\ref{D})$ & $0.102$ & $0.489$ & $0.509$ & $0.498$ & $0.179$ \\ \hline
				\multirow{2}*{$N = 100$}  & $E(z) \pm \Delta_z$ & $0.00 \pm 0.09$ & $0.0 \pm 0.7$ & $0.0 \pm 0.8$ & $0.0 \pm 0.7$ & $0.0 \pm 0.2$ \\ \cline{2-7}
										& $D(z)$ & $0.01$ & $0.494$ & $0.518$ & $0.497$ & $0.020$ \\ \hline
				\multirow{2}*{$N = 1000$} & $E(z) \pm \Delta_z$ & $0.00 \pm 0.04$ & $0.0 \pm 0.7$ & $0.0 \pm 0.7$ & $0.0 \pm 0.7$ & $0.0 \pm 0.05$ \\ \cline{2-7}
										& $D(z)$ & $0.001$ & $0.528$ & $0.501$ & $0.520$ & $0.002$\\ \hline  		
			\end{tabular}
		\caption{Характеристики выборок нормального распределения}
		\end{center}
	\end{table}

	\begin{table}[h]
		\begin{center}
			\begin{tabular}{|*{7}{c|}} \hline
				\multicolumn{7}{|c|}{Распределение Лапласа}\\ \hline
				\multicolumn{2}{|c|}{} & $\overline{x}$ & $med\; x$ & $z_R$ & $z_Q$ & $z_{tr}$ \\ \hline
				\multirow{2}*{$N = 10$}   & $E(z) \pm \Delta_z$ & $0.0 \pm 0.3$ & $0.0 \pm 0.5$ & $0.0 \pm 0.4$ & $0.0 \pm 0.4$ & $0.0 \pm 0.1$ \\ \cline{2-7}
										& $D(z)$ & $0.093$ & $0.512$ & $0.428$ & $0.479$ & $0.116$ \\ \hline
				\multirow{2}*{$N = 100$}  & $E(z) \pm \Delta_z$ & $0.0 \pm 0.1$ & $0.0 \pm 0.8$ & $0.0 \pm 0.8$ & $0.0 \pm 0.8$ & $0.0 \pm 0.2$ \\ \cline{2-7}
										& $D(z)$ & $0.01$ & $0.494$ & $0.518$ & $0.497$ & $0.020$ \\ \hline
				\multirow{2}*{$N = 1000$} & $E(z) \pm \Delta_z$ & $0.00 \pm 0.04$ & $0.0 \pm 0.7$ & $0.0 \pm 0.8$ & $0.0 \pm 0.7$ & $0.00 \pm 0.05$ \\ \cline{2-7}
										& $D(z)$ & $0.001$ & $0.492$ & $0.521$ & $0.485$ & $0.002$\\ \hline					
			\end{tabular}
			\caption{Характеристики выборок распределения Лапласа}
		\end{center}
	\end{table}
	
	\begin{table}[h]
		\begin{center}
			\begin{tabular}{|*{7}{c|}} \hline
				\multicolumn{7}{|c|}{Распределение Коши}\\ \hline
				\multicolumn{2}{|c|}{} & $\overline{x}$ & $med\; x$ & $z_R$ & $z_Q$ & $z_{tr}$ \\ \hline
				\multirow{2}*{$N = 10$}   & $E(z) \pm \Delta_z$ & $0 \pm 16$ & $0 \pm 48$ & $0 \pm 33$ & $0 \pm 14$ & $0 \pm 23$ \\ \cline{2-7}
										& $D(z)$ & $260$ & $2342$ & $1063$ & $194$ & $558$ \\ \hline
				\multirow{2}*{$N = 100$}  & $E(z) \pm \Delta_z$ & $0 \pm 12$ & $-1 \pm 22$ & $1\pm 12$ & $-2 \pm 105$ & $1 \pm 23$ \\ \cline{2-7}
										& $D(z)$ & $142$ & $499$ & $137$ & $11084$ & $492$ \\ \hline
				\multirow{2}*{$N = 1000$} & $E(z) \pm \Delta_z$ & $0 \pm 16$ & $0 \pm 34$ & $0 \pm 21$ & $2 \pm 64$ & $0 \pm 26$ \\ \cline{2-7}
										& $D(z)$ & $259$ & $1170$ & $461$ & $4212$ & $694$\\ \hline					
			\end{tabular}
			\caption{Характеристики выборок распределения Коши}
		\end{center}
	\end{table}

	\newpage

	\begin{table}[h]
		\begin{center}
			\begin{tabular}{|*{7}{c|}} \hline
				\multicolumn{7}{|c|}{Распределение Пуассона}\\ \hline
				\multicolumn{2}{|c|}{} & $\overline{x}$ & $med\; x$ & $z_R$ & $z_Q$ & $z_{tr}$ \\ \hline
				\multirow{2}*{$N = 10$}   & $E(z) \pm \Delta_z$ & $10 \pm 1$ & $10 \pm 2$ & $10 \pm 2$ & $10 \pm 2$ & $10 \pm 1$ \\ \cline{2-7}
										& $D(z)$ & $1.1$ & $4.8$ & $5.1$ & $4.7$ & $1.6$ \\ \hline
				\multirow{2}*{$N = 100$}  & $E(z) \pm \Delta_z$ & $10.0 \pm 0.3$ & $9.9 \pm 2.3$ & $10.0 \pm 2.4$ & $10.0 \pm 2.2$ & $10.0 \pm 0.5$ \\ \cline{2-7}
										& $D(z)$ & $0.092$ & $5.151$ & $4.989$ & $4.910$ & $0.199$ \\ \hline
				\multirow{2}*{$N = 1000$} & $E(z) \pm \Delta_z$ & $10.00 \pm 0.09$ & $9.8 \pm 2.3$ & $10.0 \pm 2.2$ & $10.0 \pm 2.2$ & $10.00 \pm 0.15$ \\ \cline{2-7}
										& $D(z)$ & $0.0098$ & $5.1100$ & $4.8222$ & $4.7592$ & $0.0211$\\ \hline					
			\end{tabular}
			\caption{Характеристики выборок распределения Пуассона}
		\end{center}
	\end{table}

	\begin{table}[h]
		\begin{center}
			\begin{tabular}{|*{7}{c|}} \hline
				\multicolumn{7}{|c|}{Равномерное распределение}\\ \hline
				\multicolumn{2}{|c|}{} & $\overline{x}$ & $med\; x$ & $z_R$ & $z_Q$ & $z_{tr}$ \\ \hline
				\multirow{2}*{$N = 10$}   & $E(z) \pm \Delta_z$ & $0.0 \pm 0.3$ & $0.0 \pm 0.7$ & $0.0 \pm 0.8$ & $0.0 \pm 0.8$ & $0.0 \pm 0.4$ \\ \cline{2-7}
										& $D(z)$ & $0.091$ & $0.477$ & $0.509$ & $0.506$ & $0.150$ \\ \hline
				\multirow{2}*{$N = 100$}  & $E(z) \pm \Delta_z$ & $0.0 \pm 0.1$ & $0.0 \pm 0.8$ & $0.0 \pm 0.8$ & $0.0 \pm 0.7$ & $0.0 \pm 0.2$ \\ \cline{2-7}
										& $D(z)$ & $0.103$ & $0.727$ & $0.713$ & $0.701$ & $0.146$ \\ \hline
				\multirow{2}*{$N = 1000$} & $E(z) \pm \Delta_z$ & $0.00 \pm 0.04$ & $0.0 \pm 0.7$ & $0.0 \pm 0.7$ & $0.0 \pm 0.7$ & $0.00 \pm 0.05$ \\ \cline{2-7}
										& $D(z)$ & $0.001$ & $0.500$ & $0.496$ & $0.489$ & $0.002$\\ \hline					
			\end{tabular}
			\caption{Характеристики выборок равномерного распределения}
		\end{center}
	\end{table}

\newpage

\section*{Заключение}
\addcontentsline{toc}{section}{Заключение}

В результате лабораторной работы были сгенерированы выборки размером 10, 100, 1000 элементов по заданным распределений и оценены их числовые характеристики. Работа велась в среде аналитических вычислений Maxima.

Из всех рассмотренных распределений самую большую дисперсию (на 3-4 порядка относительно других) имеет распределение Коши в связи с мощными выбросами.

Можно сделать вывод, что для данных распределений, чем больше мощность выборки, тем ближе значение медианы к своему теоретическому значению и значение выборочного среднего ближе к теоретическому матожиданию. 

\newpage

\addcontentsline{toc}{section}{Список литературы}

\begin{thebibliography}{9}

        \bibitem{theory}
        Теоретическое приложение к лабораторным работам №1-4 по дисциплине «Математическая статистика». -- СПб.: СПбПУ, 2020. -- 12 c 
	
\end{thebibliography}

\newpage

\section*{Приложение А. Репозиторий с исходным кодом}
\addcontentsline{toc}{section}{Приложение А. Репозиторий с исходным кодом}

Исходный код скрипта для среды аналитических вычислений Maxima находится в репозитории GitHub -- URL \url{https://github.com/malyarenko-md/TeorVer}

\end{flushleft}

\end{document}