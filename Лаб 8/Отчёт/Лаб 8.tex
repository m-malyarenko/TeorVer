%%%%%%%%%%%%%%%%%%%%%%%%%%%%%%%%%%%%%%%%%%%%%%%%%%%%%%%%%%%%%%%%%%%%%%%%%%%%%%%
%                     Отчёт по лабораторной работе №8
%
% Дисциплина: Теория вероятностей и Математическая статистика
%                     
% Название:   Доверительные интервалы для параметров нормального распределения
%
% Выполнил:   Михаил Маляренко
%
% Дата:       12 Янв. 2021
%
%%%%%%%%%%%%%%%%%%%%%%%%%%%%%%%%%%%%%%%%%%%%%%%%%%%%%%%%%%%%%%%%%%%%%%%%%%%%%%%

% HEADER BEGIN 
\documentclass[12pt]{article}
\usepackage[utf8]{inputenc}
\usepackage[russian]{babel}
\usepackage{pscyr}
\usepackage[T2A]{fontenc}
\usepackage{geometry}
\usepackage{graphicx}
\usepackage{multirow}
\usepackage{hhline}
\usepackage{amsmath}
\usepackage{amssymb}
\usepackage{hyperref}
\usepackage{xcolor}

\geometry {	
	a4paper, 
	left   = 20mm, 
	right  = 20mm, 
	top    = 20mm, 
	bottom = 20mm
}

\definecolor{urlcolor}{HTML}{2484BC} 
\definecolor{linkcolor}{HTML}{000000}

\graphicspath{{resource/}}
% HEADER END

% DIFINES BEGIN
\newcommand{\lskip}{\hfill\break}
% DEFINES END

\begin{document}

\begin{titlepage}
	\begin{center}
		\hfill \break
		{\textbf{Санкт-Петербургский политехнический университет Петра Великого}}\\
		\hfill \break
		\textbf{Институт прикладной математики и механики}\\
		 \hfill \break
		\textbf{Кафедра <<Телематика (при ЦНИИ РТК)>>}\\
		\vfill
		\large{\bfseries Отчет по лабораторной работе}\\
		\hfill \break
		\hfill \break
		\hfill \break
		\hfill \break
        \normalsize{\bfseriesДоверительные интервалы для парметров нормальнго распределения}\\
        \hfill \break
		По дисциплине <<Теория вероятностей и Математическая статистика>>\\
		\hfill \break
		\hfill \break
	\end{center}
 
	\normalsize
	{ 
		\begin{tabular}{lp{2cm}cr}
			Выполнил &&&\\
			Студент гр. 3630201/80101&&\underline{\hspace{1.5cm}}& М. Д. Маляренко\\\\
			Руководитель&&&\\ 
			к.ф.-м.н., доцент && \underline{\hspace{1.5cm}}& А. Н. Баженов \\\\
			&&&<<\underline{\phantom{333}}>>\underline{\phantom{сентября000}}
			2020г.
		\end{tabular}
	}
\vfill

\begin{center} Санкт-Петербург \\2021 \end{center}
\end{titlepage}

\newpage

\setcounter{page}{2}

\begin{flushleft}

\setlength{\parindent}{1cm}

% TABLE OF CONTENTS
\tableofcontents

\newpage

% LIST OF TABLES
\listoftables

\newpage

\section{Постановка задачи}

Для двух выборок размерами 20 и 100 элементов, сгенерированных согласно нормальному закону $N(x,0,1)$, для параметров положения и масштаба построить асимптотически нормальные интервальные оценки на основе точечных оценок метода максимального правдоподобия и классические интервальные оценки на основе статистик $\chi^2$ и Стьюдента. В качестве параметра надёжности взять $\gamma = 0.95$

\newpage

\section{Теория}

	\subsection{Доверительные интервалы}
		Дана выборка размером $n (x_1,\dots,x_n)$ из генеральной совокупности. Для нее построим выборочное среднее $\overline{x}$ и среднеквадратическое отклонение $s$.\cite{theory}\\
		\phantom{0}\\
		Параметры расположения $\mu$ и масштаба $\sigma$ неизвестны. Построим для них доверительный интервал с доверительной вероятностью $\gamma$.

		\subsubsection{Оценка на основе статистики Стьюдента и хи-квадрат}
			Оценка для параметра положения\cite{theory}:
			\begin{equation}
				P\left(\overline{x}-\frac{s\cdot t_{1-\alpha/2}(n-1)}{\sqrt{n-1}} < \mu < \overline{x}+\frac{s\cdot t_{1-\alpha/2}(n-1)}{\sqrt{n-1}}\right) = \gamma,
				\label{mStud}
			\end{equation}
			где $1 - \alpha = \gamma, \; t_{1-\alpha/2}(n-1)$ --- квантиль распределения Стьюдента с $(n-1)$ степенями свободы порядка $1 - \alpha/2$.\\
			\phantom{0}\\
			Оценка для параметра масштаба:
			\begin{equation}
				P\left(\frac{s\sqrt{n}}{\sqrt{\mathstrut \chi^2_{1 - \alpha/2}(n-1)}} < \sigma < \frac{s\sqrt{n}}{\sqrt{\mathstrut \chi^2_{\alpha/2}(n-1)}}\right) = \gamma,
				\label{sStud}
			\end{equation}
			где $1 - \alpha = \gamma, \; \chi^2_{p}(n-1)$ --- квантиль распределения хи-квадрат с $(n-1)$ степенями свободы порядка $p$.\\
			\phantom{0}\\
			Эти оценки справедливы для выборки из нормальной генеральной совокупности.

		\subsubsection{Асимптотические оценки на основе центральной предельной теоремы}
			Оценка для параметра положения\cite{theory}:
			\begin{equation}
				P\left(\overline{x}-\frac{s\cdot u_{1-\alpha/2}}{\sqrt{n}} < \mu < \overline{x}+\frac{s\cdot u_{1-\alpha/2}}{\sqrt{n}}\right) \approx \gamma,
				\label{mAsympt}
			\end{equation}
			где $1-\alpha = \gamma, u_{1-\alpha/2}$ --- квантиль стандартного нормального распределения порядка $1 - \alpha/2$.\\
			\phantom{0}\\
			Для оценки параметра масштаба необходимо рассчитать выборочный эксцесс $e = \frac{m_4}{s^4}-3$, где $m_4 = \frac{1}{n}\sum(x_i - \overline{x})^4$ --- четвёртый выборочный центральный момент.\\
			\phantom{0}\\
			Парметр масштаба можно оценить так:
			\begin{equation}
				P\left(s(1 + U)^{-0.5} < \sigma < s(1 - U)^{-0.5}\right) \approx \gamma,
				\label{sAsympt}
			\end{equation} 
			где $U = u_{1-\alpha/2}\sqrt{\mathstrut (e+2)/n}, \; u_{1-\alpha/2}$ --- квантиль стандартного нормального распределения порядка $1 - \alpha/2$.\\
			\phantom{0}\\
			Эти оценки справедливы для выборки из генеральной совокупности, которая имеет конечные центральные моменты вплоть до 4 порядка и конечное матожидание.
\newpage

\section{Реализация}

	Расчёты реализованы в среде аналитических вычислений Maxima. Сначала были сгенерированные две нормально распределённые выборки размера 20 и 100. Далее для их вычислялось среднеквадратическое отклонение, средневыборочная величина. Затем происходили вычисления по формулам (\ref{mStud}) (\ref{sStud}) (\ref{mAsympt}) (\ref{sAsympt}). Для расчётов использовались встроенные статистические функции из пакета \texttt{descriptive}. Полный текст скрипта представлен в репозитории на GitHub.

\newpage

\section{Результаты}

	В Таблице \ref{interv_chi} представлены оценки параметров нормального распределения на основе статистик Стьюдента и хи-квадрат.

		\begin{table}[h]
			\begin{center}
				\caption{Интервальные оценки на основе статистик Стьюдента и хи-квадрат}
				\begin{tabular}{||c||c|c||} \hhline{~|t:=:=:t|}
					\multicolumn{1}{c||}{} & $\mu$ (\ref{mStud}) & $\sigma$ (\ref{sStud})\\
					\hhline{|t:=::=:=:|}
					$N = 20$ & $-0.34 < \mu < 0.63$ & $0.79 < \sigma < 1.50$\\
					\hhline{||-||-|-||}
					$N = 100$ & $-0.17 < \mu < 0.21$ & $0.84 < \sigma < 1.12$\\
					\hhline{|b:=:=:=:b|}
				\end{tabular}
			\label{interv_chi}
			\end{center}
		\end{table}

		В Таблице \ref{interv_asumpt} представлены асимптотические интервальные оценки параметров нормального распределения.

		\begin{table}[h]
			\begin{center}
				\caption{Асимптотические интервальные оценки}
				\begin{tabular}{||c||c|c||} \hhline{~|t:=:=:t|}
					\multicolumn{1}{c||}{} & $\mu$ (\ref{mAsympt}) & $\sigma$ (\ref{sAsympt})\\
					\hhline{|t:=::=:=:|}
					$N = 20$ & $-0.29 < \mu < 0.59$ & $0.82 < \sigma < 1.41$\\
					\hhline{||-||-|-||}
					$N = 100$ & $-0.04 < \mu < 0.35$ & $0.86 < \sigma < 1.09$\\
					\hhline{|b:=:=:=:b|}
				\end{tabular}
			\label{interv_asumpt}
			\end{center}
		\end{table}

\newpage

\section*{Заключение}
\addcontentsline{toc}{section}{Заключение}

	В результате выполнения лабораторной работы были построены доверительные интервалы для параметров закона распределения выборок размера 20 и 100. Как видно из полученных результатов (Таблицы \ref{interv_chi}, \ref{interv_asumpt}) асимптотический метод оценки доверительного интервала показал более точные результаты по сравнения оценок на основе статистик Стьюдента и хи-квадрат.

\newpage

\addcontentsline{toc}{section}{Список литературы}
\begin{thebibliography}{9}

	\bibitem{theory}
    Теоретическое приложение к лабораторным работам №5-8 по дисциплине «Математическая статистика». -- СПб.: СПбПУ, 2020. -- 22 c 
    
    \bibitem{maks}
    Максимов Ю.Д. Математика. Теория и практика по математической статистике. Конспект-справочник по теории вероятностей : учеб. пособие / Ю.Д. Максимов; под ред. В.И. Антонова. — СПб. : Изд-во Политехн. ун-та, 2009. — 395 с. (Математика в политехническом университете).

\end{thebibliography}

\newpage

\section*{Приложение А. Репозиторий с исходным кодом}
\addcontentsline{toc}{section}{Приложение А. Репозиторий с исходным кодом}

Исходный код скрипта для среды аналитических вычислений Maxima находится в репозитории GitHub -- URL \url{https://github.com/malyarenko-md/TeorVer}

\end{flushleft}

\end{document}